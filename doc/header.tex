\usepackage{lipsum}
\usepackage[colorinlistoftodos,
textsize=footnotesize,textwidth=0.90\marginparwidth]{todonotes}
\usepackage{lineno}
\usepackage{setspace}
\usepackage[parfill]{parskip}

%%%%%%%%%%%%%%%%%%%%%%%%%%%%%%%%%%%%%%%%%%%%%%
% new command for inserting initialed comments
\newcommand{\njg}[1]{\todo[color=blue!20]{\textbf{njg}: #1}} 

% new command for inserting needed references
\newcommand{\addref}[1]{\todo[color=red!40]{\textsc{add ref}: \\ #1}}
%%%%%%%%%%%%%%%%%%%%%%%%%%%%%%%%%%%%%%%%%%%%%%%%
%add the following to the YAML of the markdown document. Some of this is created in Rstudio, but other lines are not:

%csl: ecology.csl
%output:
%  pdf_document:
%    latex_engine: xelatex
%    number_sections: yes
%    toc_depth: 3
%    keep_tex: true
%    includes:
%      in_header: header.tex
%bibliography: eco-centennial-paper.bib   

%%%%%%%%%%%%%%%%%%%%%%%%%%%%%%%%%%%%%%%%%%%%%%%%
% common commands to use in the .Rmd file with these packages

% \lipsum[1-5] 
%for 5 paragraphs of text filler

% \doublespacing 
%for double-spaced manuscript

%\setlength{\parindent}{0cm}
%for indenting of paragraphs

%\setlength{\parskip}{2ex plus4mm minus3mm}
%for amount of spacing between paragraphs

%to start line numbering
%\linenumbers

%\listoftodos
%for generating a list of all the notes
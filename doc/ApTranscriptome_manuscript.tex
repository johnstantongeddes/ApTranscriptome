\documentclass[]{article}
\usepackage{lmodern}
\usepackage{amssymb,amsmath}
\usepackage{ifxetex,ifluatex}
\usepackage{fixltx2e} % provides \textsubscript
\ifnum 0\ifxetex 1\fi\ifluatex 1\fi=0 % if pdftex
  \usepackage[T1]{fontenc}
  \usepackage[utf8]{inputenc}
\else % if luatex or xelatex
  \ifxetex
    \usepackage{mathspec}
    \usepackage{xltxtra,xunicode}
  \else
    \usepackage{fontspec}
  \fi
  \defaultfontfeatures{Mapping=tex-text,Scale=MatchLowercase}
  \newcommand{\euro}{€}
\fi
% use upquote if available, for straight quotes in verbatim environments
\IfFileExists{upquote.sty}{\usepackage{upquote}}{}
% use microtype if available
\IfFileExists{microtype.sty}{%
\usepackage{microtype}
\UseMicrotypeSet[protrusion]{basicmath} % disable protrusion for tt fonts
}{}
\usepackage[margin=1in]{geometry}
\usepackage{longtable,booktabs}
\usepackage{graphicx}
\makeatletter
\def\maxwidth{\ifdim\Gin@nat@width>\linewidth\linewidth\else\Gin@nat@width\fi}
\def\maxheight{\ifdim\Gin@nat@height>\textheight\textheight\else\Gin@nat@height\fi}
\makeatother
% Scale images if necessary, so that they will not overflow the page
% margins by default, and it is still possible to overwrite the defaults
% using explicit options in \includegraphics[width, height, ...]{}
\setkeys{Gin}{width=\maxwidth,height=\maxheight,keepaspectratio}
\ifxetex
  \usepackage[setpagesize=false, % page size defined by xetex
              unicode=false, % unicode breaks when used with xetex
              xetex]{hyperref}
\else
  \usepackage[unicode=true]{hyperref}
\fi
\hypersetup{breaklinks=true,
            bookmarks=true,
            pdfauthor={},
            pdftitle={Thermal reactionomes reveal divergent responses to thermal extremes in warm and cool-climate ant species},
            colorlinks=true,
            citecolor=blue,
            urlcolor=blue,
            linkcolor=magenta,
            pdfborder={0 0 0}}
\urlstyle{same}  % don't use monospace font for urls
\setlength{\parindent}{0pt}
\setlength{\parskip}{6pt plus 2pt minus 1pt}
\setlength{\emergencystretch}{3em}  % prevent overfull lines
\setcounter{secnumdepth}{0}

%%% Use protect on footnotes to avoid problems with footnotes in titles
\let\rmarkdownfootnote\footnote%
\def\footnote{\protect\rmarkdownfootnote}

%%% Change title format to be more compact
\usepackage{titling}

% Create subtitle command for use in maketitle
\newcommand{\subtitle}[1]{
  \posttitle{
    \begin{center}\large#1\end{center}
    }
}

\setlength{\droptitle}{-2em}
  \title{Thermal reactionomes reveal divergent responses to thermal extremes in
warm and cool-climate ant species}
  \pretitle{\vspace{\droptitle}\centering\huge}
  \posttitle{\par}
  \author{}
  \preauthor{}\postauthor{}
  \date{}
  \predate{}\postdate{}

\usepackage{lipsum}
\usepackage[colorinlistoftodos,
textsize=footnotesize,textwidth=0.90\marginparwidth]{todonotes}
\usepackage{lineno}
\usepackage{setspace}
\usepackage[parfill]{parskip}

%%%%%%%%%%%%%%%%%%%%%%%%%%%%%%%%%%%%%%%%%%%%%%
% new command for inserting initialed comments
\newcommand{\njg}[1]{\todo[color=blue!20]{\textbf{njg}: #1}} 

% new command for inserting needed references
\newcommand{\addref}[1]{\todo[color=red!40]{\textsc{add ref}: \\ #1}}
%%%%%%%%%%%%%%%%%%%%%%%%%%%%%%%%%%%%%%%%%%%%%%%%
%add the following to the YAML of the markdown document. Some of this is created in Rstudio, but other lines are not:

%csl: ecology.csl
%output:
%  pdf_document:
%    latex_engine: xelatex
%    number_sections: yes
%    toc_depth: 3
%    keep_tex: true
%    includes:
%      in_header: header.tex
%bibliography: eco-centennial-paper.bib   

%%%%%%%%%%%%%%%%%%%%%%%%%%%%%%%%%%%%%%%%%%%%%%%%
% common commands to use in the .Rmd file with these packages

% \lipsum[1-5] 
%for 5 paragraphs of text filler

% \doublespacing 
%for double-spaced manuscript

%\setlength{\parindent}{0cm}
%for indenting of paragraphs

%\setlength{\parskip}{2ex plus4mm minus3mm}
%for amount of spacing between paragraphs

%to start line numbering
\linenumbers

%\listoftodos
%for generating a list of all the notes


\begin{document}

\maketitle


\linenumbers

\newpage

\textbf{Authors}

John Stanton-Geddes \textsuperscript{1}, Andrew Nguyen
\textsuperscript{1}, Lacy Chick \textsuperscript{2}, James Vincent
\textsuperscript{3}, Mahesh Vangala \textsuperscript{3}, Robert R. Dunn
\textsuperscript{4}, Aaron M. Ellison \textsuperscript{5}, Nathan J.
Sanders \textsuperscript{2,6}, Nicholas J. Gotelli \textsuperscript{1},
Sara Helms Cahan \textsuperscript{1}

\vspace{10 mm}

\textbf{Affiliations}

\textsuperscript{1} Department of Biology, University of Vermont,
Burlington, Vermont 05405 \newline

\textsuperscript{2} Department of Ecology and Evolutionary Biology,
University of Tennessee, Knoxville, Tenessee 37996 \newline

\textsuperscript{3} Vermont Genetics Network, University of Vermont,
Burlington, Vermont 05405 \newline

\textsuperscript{4} Department of Biological Sciences, North Carolina
State University, Raleigh, North Carolina 27695 \newline

\textsuperscript{5} Harvard Forest, Harvard University, Petersham,
Massachusetts 01336 \newline

\textsuperscript{6} Center for Macroecology, Evolution and Climate,
University of Copenhagen, Universitetsparken 15, DK-2100 Copenhagen

\textbf{Corresponding author}

John Stanton-Geddes, Data Scientist

Dealer.com, 1 Howard St, Burlington, VT 05401

Phone: 802-656-2922

Email:
\href{mailto:john.stantongeddes.research@google.com}{\nolinkurl{john.stantongeddes.research@google.com}}

\newpage

\section{Abstract}\label{abstract}

\textbf{Background} The distributions of species and their responses to
climate change are in part determined by their thermal tolerances.
However, little is known about how thermal tolerance evolves. To test
whether evolutionary extension of thermal limits is accomplished through
enhanced cellular stress response (\emph{enhanced response}),
constitutively elevated expression of protective genes (\emph{genetic
assimilation}) or a shift from damage resistance to passive mechanisms
of thermal stability (\emph{tolerance}), we conducted an analysis of the
\emph{reactionome}: the reaction norm for all genes in an organism's
transcriptome measured across an experimental gradient. We characterized
thermal reactionomes of two common ant species in the eastern U.S, the
northern cool-climate \emph{Aphaenogaster picea} and the southern
warm-climate \emph{Aphaenogaster carolinensis}, across 12 temperatures
that spanned their entire thermal breadth.

\textbf{Results} We found that at least 2\% of all genes changed
expression with temperature. The majority of upregulation was specific
to exposure to low temperatures. The cool-adapted \emph{A. picea}
induced expression of more genes in response to extreme temperatures
than did \emph{A. carolinensis}, consistent with the \emph{enhanced
response} hypothesis. In contrast, under high temperatures the
warm-adapted \emph{A. carolinensis} downregulated many of the genes
upregulated in \emph{A. picea}, and required more extreme temperatures
to induce down-regulation in gene expression, consistent with the
\emph{tolerance} hypothesis. We found no evidence for a trade-off
between constitutive and inducible gene expression as predicted by the
\emph{genetic assimilation hypothesis}.

\textbf{Conclusions} These results suggest that increases in upper
thermal limits may require an evolutionary shift in response mechanism
away from damage repair toward tolerance and prevention.

\textbf{Keywords}

\emph{Aphaenogaster}, gene expression, plasticity, reactionome,
transcriptome

\section{Background}\label{background}

Temperature regulates biological activity and shapes diversity from
molecular to macroecological scales {[}1, 2{]}. Many species, especially
small-bodied arthropods, live at temperatures close to their thermal
limits and are at risk from current increases in temperature {[}3--5{]}.
Thermal tolerance, the ability of individuals to maintain function and
survive thermal extremes, depends on a complex interplay between the
structural integrity of cellular components and activation of
physiological response mechanisms to prevent and/or repair damage {[}6,
7{]}. Thermal defense strategies are shaped by the environmental regime
organisms experience {[}8{]} and thermal limits vary considerably among
species and populations {[}3, 4, 9, 10{]}. These differences in thermal
tolerance are largely genetic {[}11, 12{]} with a highly polygenic basis
{[}13--16{]}. Outside of candidate genes {[}13{]}, little is known about
the evolution of thermal tolerance or the link between short-term
physiological acclimation and longer-term adaptation to novel
temperature regimes. This information is critical for understanding the
adaptive potential of species to future climates {[}17{]}.

To address this gap of knowledge, we need information on the extent to
which selection has acted upon the diversity and plasticity of genes
involved in thermal tolerance {[}17, 18{]}. In recent years,
whole-organism gene expression approaches (e.g.~transcriptomics) using
high-throughput RNA sequencing (RNAseq) technology have been widely
applied to identify genes involved in thermal tolerance {[}19--22{]} and
other traits. Such studies typically use an ANOVA-type experimental or
sampling design, with only a few environmental levels, and often find
only a few dozen to hundred genes with differential expression in
different thermal regimes. However, temperature and other environmental
factors vary continuously in nature. As a result, such categorical
comparisons (e.g.~high vs.~low temperatures) are likely to miss key
differences that are due not just to whether it is hot, but rather how
hot it is. Continuous variation is better characterized with a reaction
norm approach, which describes variation in the phenotype of a single
genotype across an environmental gradient {[}23{]}. Reaction norms
differ not only in mean values, but also in their shapes {[}10, 24{]},
and differences in the shape of reaction norms are often larger than
differences in mean values at both the species and the population level
{[}24{]}.

In this study, we extend the reaction norm approach to RNAseq analysis
and introduce the \emph{reactionome}, which we define as a
characterization of the reaction norm for all genes in an organism's
transcriptome across an environmental gradient. Although temporal
patterns of transcriptional activity (e.g.~fast- vs.~slow- responding
genes) are also important components of an organism's transcriptional
response to environmental conditions {[}25{]}, we focus here on the
response of transcripts across conditions at the same time point.

We used the reactionome method to identify genes that are thermally
responsive in two closely-related eastern North American ant species,
\emph{Aphaenogaster carolinensis} and \emph{A. picea} {[}26, 27{]}.
\emph{Aphaenogaster} are some of the most common ants in eastern North
America {[}28{]} where they are keystone seed dispersers {[}29--31{]}.
Ants, and ectotherms in general, have little or no thermal safety margin
{[}5{]} and thus are highly susceptible to climate change {[}4, 32{]},
putting at risk important ecosystem services {[}33{]}. Growth chamber
studies have demonstrated that reproduction of \emph{Aphaenogaster} will
be compromised by increased tempreatures {[}34{]}, while field studies
{[}32{]} and simulations {[}35{]} indicate that ant species persistence
will depend on combinations of physiology and species interactions.
\emph{Aphaenogaster carolinensis} experiences a higher mean annual
temperature (MAT) (14.6°C) and less seasonal temperature variation
(temperature seasonality = 7,678°) than does \emph{A. picea} (MAT =
4.6°C, seasonality = 10,008°; {[}36{]}) at their respective collection
sites. In controlled laboratory experiments, these warm- and
cold-climate species exhibit corresponding differences in their critical
maximum (44.7°C for \emph{A. carolinensis} versus 41.3°C for \emph{A.
picea}; see Methods) and minimum temperatures (6.1°C for \emph{A.
carolinensis} versus -0.1°C for \emph{A. picea}). These differences
between species in their thermal environments and physiological
tolerances allowed us to investigate adaptation to both lower and upper
thermal extremes in this system.

To characterize the thermal reactionome, we measured the reaction norm
for each gene using a regression-based statistical approach to identify
temperature-dependent patterns of change in gene expression. We used
these response patterns to quantitatively test three mechanistic
hypotheses of thermal adaptation. First, the \emph{enhanced response
hypothesis} {[}37--39{]} proposes that species extend their thermal
limits through a stronger induced response to provide greater protection
from more frequently encountered stressors. This hypothesis would
predict that the cool-adapted \emph{A. picea} would activate more genes,
and induce them more strongly, in response to low temperatures than
would the warm-adapted \emph{A. carolinensis}, which would show greater
induction in response to high temperatures.

Second, the \emph{tolerance hypothesis} {[}9, 40{]} proposes that
existing inducible stress responses become insufficient or prohibitively
costly as environmental stressors increase in frequency, resulting in a
shift away from an induced response in favor of structural changes
{[}41{]} or behavioral adaptations {[}5, 42{]}. This hypothesis predicts
adaptation to stress should be associated with lower transcriptional
responsiveness and less sensitivity to temperature perturbation, as well
as a shift to an alternate suite of tolerance genes and pathways {[}43,
44{]}.

Finally, the \emph{genetic assimilation hypothesis} {[}45, 46{]}
proposes that exposure to more extreme stressors selects for a shift
from inducible to constitutive expression of stress-response genes. This
hypothesis predicts that transcripts responsive to high temperatures in
\emph{A. picea} will have higher constitutive expression in \emph{A.
carolinensis}, whereas transcripts responsive to low temperatures in
\emph{A. carolinensis} will have higher constitutive expression in
\emph{A. picea}.

To summarise, in this project we generated the transcriptomes of two
closely-related temperate ant species, and quantified their gene
expression across a wide range of thermal conditions. We then evaluated
three non-mutually exclusive hypotheses (enhanced response, tolerance
and genetic assimilation) of the evolution of thermal adaptation by
comparing the number and expression patterns of transcripts between
species in response to extreme low and extreme high temperatures.
Finally, we used gene ontology information to determine which gene
products and pathways are involved in thermal adaptation in the two
species.

\section{Results}\label{results}

\subsection{Reaction norms of thermally-responsive
transcripts}\label{reaction-norms-of-thermally-responsive-transcripts}

The combined \emph{Aphaenogaster} transcriptome assembly contained
99,861 transcripts. About half of these (51,246) transcripts had a
signficant BLAST hit, of which 50\% (25,797) had a top hit to Insecta
and 37\% (18,854) had a top hit to Formicidae. We performed a BUSCO
analysis {[}47{]} to assess the quality of the transcriptome assembly
against the arthropod Benchmarking Universal Single-Copy Orthologs
(BUSCOs). This analysis revealed that transcriptome is largely complete,
as we recovered 1,426 complete single-copy BUSCOs (62\%) and an
additional 435 fragmented BUSCOs (16\%), which is in line with results
of Simao et al. {[}47{]} for transcriptomes of other non-model species.
Moreover, only 8\% of the BUSCOs were found to be duplicated in the
transcriptome, which indicates that the steps (see Methods) we took to
collapse homologs in the combined transcriptome of the two species were
successful.

We quantified gene expression using the program \texttt{Sailfish}
{[}48{]}, and fitted polynomial regression models to the expression
values of each transcript to identify those that had a linear or
quadratic relationship (Fig. 1). To account for multiple tests, we both
applied a False Discovery Rate (FDR) correction, and performed a
resampling analysis to determine the number of transcripts that would be
expected to have a significant relationship by chance alone. We retained
the 2,509 (2.5\% of total) transcripts that exceeded the null
expectation from the resampling analysis as true positive transcripts
for further analyses (Table S1). Of these transcripts, 75\% (1,553) had
a non-linear relationship with temperature that would likely have been
missed with a standard differential expression experiment (e.g.~high
vs.~low temperature). The proportion of responsive transcripts is
similar if we focus only on those transcript with a BLAST hit (725
significant transcripts out of 51,246, 1.4). However, as with all
\emph{de novo} transcriptome assemblies, this assembly is fragmented due
to partial contigs and alternative transcripts {[}49{]} so this estimate
is likely a lower bound for the true proportion of transcripts that are
thermally responsive.

We used the predicted transcript expression levels to partition
transcripts for each species into five expression categories (Fig. 1)
which were defined \emph{a priori} to allow us to test predictions
derived from three thermal adaptation hypotheses of relative response
severity in the two species: \textbf{High} transcripts had greatest
expression at temperatures \textgreater{} 31°C, \textbf{Low} transcripts
had greatest expression at temperatures \textless{} 10°C,
\textbf{Intermediate} transcripts had greatest expression between 10 to
30°C, \textbf{Bimodal} transcripts had increased expression at both high
and low temperatures, while \textbf{NotResp} transcripts were those that
were not thermally responsive in the focal species but did respond in
the other.

\subsection{Expression response to thermal extremes differs between
species}\label{expression-response-to-thermal-extremes-differs-between-species}

Although the total number of thermally-responsive transcripts did not
differ between species (\(\chi^2\)\textsubscript{1} = 0.08, \emph{P} =
0.77), the two species differed in the number of transcripts in each
expression category (Table 1, \(\chi^2\)\textsubscript{4} = 302.896,
\emph{P} \textless{} 0.001). \emph{Aphaenogaster picea} induced
significantly more transcripts in response to both temperature extremes
(\textbf{Bimodal} transcripts in Table 1; \(\chi^2\)\textsubscript{1} =
71.617, \emph{P} \textless{} 0.001) than did \emph{A. carolinensis},
which downregulated more transcripts under these conditions
(\textbf{Intermediate} transcripts in Table 1;
\(\chi^2\)\textsubscript{1} = 256.329, \emph{P} \textless{} 0.001).
Consistent with the \emph{enhanced response} hypothesis, the
cool-climate \emph{A. picea} induced 273 (\textasciitilde{}50\%) more
transcripts in response to low temperatures than the warm-climate
\emph{A. carolinensis} (\textbf{Low} transcripts in Table 1;
\(\chi^2\)\textsubscript{1} = 71.227, \emph{P} \textless{} 0.001).
However, there was no difference among species in the number of
transcripts upregulated at high temperatures (\textbf{High} transcripts
in Table 1; \(\chi^2\)\textsubscript{1} = 0.53, \emph{P} = 0.47).

In addition, we also examined the specific patterns of shifts from one
expression category to another between species. As transcripts may
change expression between species due to neutral drift alone, we used
the Stuart-Maxwell test of marginal homogeneity to test if the number of
responsive transcripts in each expression category differed between the
species when controlling for overall differences in the number of
responsive transcripts. We found that the expression categories of
individual transcripts between the two species were not randomly
distributed (Stuart-Maxwell test of marginal homogeneity
\(\chi^2\)\textsubscript{4} = 319, P \textless{} 0.001, Fig. S1).
Specifically, the two species differed significantly in expression
pattern, which captures differences in slope as well as category, for
1,553 (62\%) of the thermally responsive transcripts.

The \emph{enhanced response} and \emph{tolerance} hypotheses make
opposing predictions concerning the overlap in response patterns between
the two species (Fig. 2). The \emph{enhanced response hypothesis} posits
that temperature adaptation uses existing mechanisms for thermal
resistance, which should result in significant overlap in response and
fewer transcripts shifting expression categories than expected by chance
(Fig. 2, left). In contrast, the \emph{tolerance hypothesis} predicts
that transcripts involved in active defense will become non-responsive
or shift to other expression categories in the better-adapted species
(Fig. 2, right). We tested these predictions by examining if the
transcripts upregulated in response to the temperature extreme
experienced less frequently by a species (cool temperatures for the
warm-climate \emph{A. carolinensis}, and warm temperatures for the
cool-climate \emph{A. picea}) displayed the same response profile in the
other species that more frequently experiences those conditions.

Transcripts upregulated at low temperatures in \emph{A. carolinensis}
(\textbf{Low} and \textbf{Bimodal} transcripts) were significantly
biased toward this same category and away from other expression
categories in \emph{A. picea} (Fig. 3A), suggesting shared response
pathways as predicted by the \emph{enhanced response} hypothesis. In
contrast, transcripts upregulated in response to high temperatures in
\emph{A. picea} (\textbf{High} and \textbf{Bimodal}) shifted expression
categories in \emph{A. carolinensis} (Fig. 3B), primarily to the
\textbf{Intermediate} category (Fig. 3B). These transcripts are less
likely to be upregulated in any context, consistent with the
\emph{tolerance hypothesis}.

\subsection{Molecular processes suggest a generalized stress response
mechanism}\label{molecular-processes-suggest-a-generalized-stress-response-mechanism}

The gene set enrichment analysis revealed a number of gene groups
enriched in each expression category (Table S2). Across both species,
there were 9 terms enriched in the \textbf{Bimodal} category, including
terms involved in stress response (regulation of cellular response to
stress, signal transduction by p53 class mediator), cell death
(apoptotic signaling pathway) and cellular organization (e.g.~protein
complex localization). The 6 terms enriched in the \textbf{Low} category
suggest that proteins undergo structural (e.g protein acylation) and
organizational (single-organism organelle organization) changes to
tolerate colder temperatures, possibly to maintain membrane fluidity
{[}50{]}. The \textbf{High} category included only a single enriched GO
term, ``nicotinamide metabolic process'', while the
\textbf{Intermediate} category had 5 terms including DNA packaging and
metabolic process terms.

\subsection{\emph{A. carolinensis} has greater inertia of expression
change to increases in temperature than does \emph{A.
picea}}\label{a.-carolinensis-has-greater-inertia-of-expression-change-to-increases-in-temperature-than-does-a.-picea}

As an additional test of the \emph{tolerance hypothesis}, we examined
the critical temperature of gene induction in response to increasing and
decreasing temperatures. We compared between species the mean
temperatures of transcript upregulation, defined as the temperature at
which the transcript showed the greatest positive change in expression.
In support of the \emph{enhanced response} but not the \emph{tolerance
hypothesis}, the temperature of induction at low temperatures was
significantly higher for the cool-climate \emph{A. picea} than for
\emph{A. carolinensis} (12.4°C) than \emph{A. picea} (13.1°C;
\emph{t}\textsubscript{1308} = -3.1, \emph{P} \textless{} 0.002; Fig.
4A), though the temperature of induction did not differ between species
for high temperatures (\emph{t}\textsubscript{567} = 0.8, \emph{P}
\textless{} 0.403).

Similarly, for down-regulated (\textbf{Intermediate}) transcripts, we
compared the mean temperatures of downregulation of transcript
expression between species at both high (\textgreater{} 20°C) and low
(\textless{} 20°C) temperatures. Consistent with the \emph{tolerance
hypothesis}, \emph{A. carolinensis} had greater inertia of gene
expression in response to increasing temperatures. The temperature of
downregulation for \textbf{Intermediate} transcripts was 28.6°C for
\emph{A. carolinensis} compared to 27.2 for \emph{A. picea}
(\emph{t}\textsubscript{294} = 3.8, \emph{P} \textless{} 0.001). The
difference between species was not significant with decreasing
temperatures (\emph{t}\textsubscript{251} = 0.5, \emph{P} = 0.584, Fig.
4B).

\subsection{No evidence for \emph{genetic
assimilation}}\label{no-evidence-for-genetic-assimilation}

We tested the \emph{genetic assimilation hypothesis} by comparing the
log ratios of relative inducibility to relative baseline expression at
the rearing temperature (25°C). If stress-response transcripts have
shifted between species from inducible to constitutive expression, there
should be a negative relationship between the two. We found no evidence
of such a relationship for either temperature extreme: transcripts more
upregulated at high temperatures in the cool-climate \emph{A. picea}
were not expressed at higher baseline levels in the warm-climate
\emph{A. carolinensis} (Fig. 5A). Similarly, transcripts more
upregulated at low temperatures in \emph{A. carolinensis} did not show
higher baseline levels in \emph{A. picea} (Fig. 5B). In fact, for both
comparisons we found a weakly positive relationship between relative
inducibility and baseline expression between the two species
(\(\beta_1\) = 0.31, \emph{P} \textless{} 0.001 and (\(\beta_1\) = 0.21,
\emph{P} \textless{} 0.001). In addition, the thermally responsive
transcripts in \emph{A. carolinensis}, regardless of expression pattern,
had higher baseline expression than those in \emph{A. picea}, including
those with \textbf{Intermediate} expression profiles in both species
(Wilcoxon V = 68842, \emph{P} \textless{} 0.001). An important exception
to this pattern is the set of transcripts that had \textbf{High} or
\textbf{Bimodal} expression in \emph{A. picea} but were not thermally
responsive in \emph{A. carolinensis} (top-row of Fig. 3B). These
transcripts are less likely to be upregulated in any context, consistent
with the \emph{tolerance hypothesis}.

\section{Discussion}\label{discussion}

The potential for many species to persist in face of climate change will
depend in part upon their thermal tolerances. However, for most species
little is known about how plasticity or adaptive changes in gene
expression underlie thermal tolerance. By using a \emph{reactionome}
approach, we were able to quantitatively describe plasticity in
transcript expression across a thermal gradient, and identify putative
changes in gene expression associated with shifts in thermal tolerance
between the ant species \emph{Aphaenogaster picea} and \emph{A.
carolinensis}. We found non-linear patterns of gene expression changes
in response to temperature, with both quantitative and qualitative
differences between species, consistent with different mechanisms of
thermal adaptation to low and high temperature extremes.

Under the \emph{enhanced response} hypothesis, stress-adapted species
are hypothesized to induce a stronger and earlier response to extreme
conditions. We found evidence for this hypothesis at low temperatures:
although the lower thermal limit for \emph{A. picea} is substantially
lower than \emph{A. carolinensis}, \emph{A. picea} upregulated
responsive transcripts at slightly less extreme temperatures (Fig. 4A).
Moreover, the transcripts upregulated in \emph{A. picea} included about
half (55\%) those upregulated in \emph{A. carolinensis} as well as an
additional set of 261 transcripts (Table 1), enriched for metabolism,
organization and translation processes (Table S2). Two non-mutually
exclusive hypotheses may explain this pattern. First, surviving
prolonged low temperatures, such as would be experienced during
overwintering, generally requires advance production of specialized
cryoprotectants {[}43{]} and a suite of preparatory physiological
modifications {[}51{]}. The northern species \emph{A. picea} may induce
a greater response to survive the longer winter period. Alternatively,
the response to low temperatures may reflect countergradient expression
to counteract reduction in enzyme efficiency, and maintain activity as
temperature declines {[}41{]}. This requirement may be under stronger
selection in \emph{A. picea} given the shorter growing season that would
necessitate foraging under a broader range of temperatures.

In contrast to cold tolerance, the enhanced upper thermal limit in
\emph{A. carolinensis} is best explained by the \emph{tolerance}
hypothesis. High temperatures were associated with significantly fewer
upregulated transcripts in \emph{A. carolinensis} (Table 1), and a large
proportion (25\%) of the transcripts upregulated at high temperatures in
\emph{A. picea} were either downregulated or expressed at negligible
levels overall in \emph{A. carolinensis}. These results suggest that
mechanisms other than the heat shock response are acting to maintain
protein stability in face of temperature increases. Such mechanisms may
include novel constitutive defenses {[}19, 21, 22{]}, enhanced proteome
stability {[}52{]} or behavioral quiescence {[}5{]} to tolerate thermal
stress. These differences are in line with expectations that \emph{A.
carolinensis}, with a growing season over twice the length of its
northern congener, may be better able to afford to restrain from
foraging in suboptimal conditions. Indeed, quiescence under stressful
conditions by the red harvester ant \emph{Pogonomyrmex barbatus} has
been shown to increase colony fitness {[}42{]}.

The one hypothesis that did not receive support was the \emph{genetic
assimilation hypothesis}, which predicts that exposure to more frequent
stressors will select for a shift from inducible to constitutive
expression of stress-response transcripts. This constrasts with other
recent studies on adaptation in field populations to thermal stress
{[}21{]}. However, in a short-term selection experiment for heat
tolerance, Sikkink et al. {[}46{]} also found no evidence for genetic
assimilation at the expression level after 10 generations of selection
for heat tolerance in \emph{Caenorhabditis remanei}, even though there
was a substantial increase in heat tolerance. Both the genetic
assimilation and tolerance routes to increasing thermal limits are
functionally similar in that they emphasize damage prevention rather
than repair. Whether a particular taxon evolves one strategy over
another may be related to availability of alternative mechanisms as well
as the intensity, frequency and duration of temperature stress in a
given environment.

Given the differences in the patterns of thermal responsiveness between
species (Fig. 3), it is worth noting a number of similarities. In both
species, there were 2 -- 3 times more transcripts upregulated at low
than high temperatures (Table 1). The degree of upregulation at low
temperatures is surprising given previous studies {[}53, 54{]} that
found little transcriptional activity at low temperatures. However,
these studies exposed organisms to a few extreme (-10 -- 0°C)
temperatures. At these extremes, we also found few upregulated
transcripts (Fig. 4A), whereas the peak of low-temperature
transcriptional activation occurred near 10°C (Fig. 4). A potential
explanation for this pattern is that increased gene expression functions
to support elevated metabolism at moderately cold temperatures, as
suggested by the metabolic cold adaptation hypothesis {[}55{]}. The
observation that more transcripts were upregulated at low than high
temperatures could also be due to stronger selection on upper than lower
thermal limits, thereby reducing both genetic variation and gene
expression plasticity at high temperatures {[}4, 56{]}. This explanation
is consistent with the observation in \emph{Aphaenogaster rudis}
{[}57{]} and other ectotherms {[}10, 58{]} that critical maximum
temperatures vary less among taxa than do critical minimum temperatures.

Critical maximum and minimum temperatures are hypothesized to be
genetically correlated {[}10, 58{]}, but this was not evident in terms
of gene expression in this study. Only \textasciitilde{}10\% of
transcripts upregulated in response to temperature were bimodal, and for
both activation and down-regulation, thresholds differed between species
at only one temperature extreme (Fig. 4). This suggests that species do
not face a fundamental trade-off between these two limits and may be
able to shift upper and lower thermal limits independently to match
requirements of more seasonally variable environments. A major
contribution of this study is the construction of a reactionome for gene
expression data. Similar approaches have been used in other species
{[}59, 60{]}, but to our knowledge, none have applied a regression
approach to identify a complete list of responsive transcript across an
environmental gradient. This approach revealed quantitative patterns of
temperature response not captured in categorical comparisons. For
example, the degree of upregulation at cool (\textasciitilde{}10°C) but
not extreme cold temperatures was missed in previous studies that
focused on extreme cold limits, as discussed above. Further, a number of
issues have hampered RNA-seq studies to date. Namely, lists of
differentially expressed transcripts are prone to false positives
{[}61{]}, depend on the genetic background of the organism {[}62{]} and
are prone to ``storytelling'' interpretations {[}63{]}. Our findings are
robust to these issues as we focus on the average change in the shape of
the reaction norms across many hundreds of responsive transcripts in
each species. Although we use gene ontology information to interpret our
results, the key findings about differential plasticity of expression
between species do not depend on functional annotation.

Moreover, by characterizing responses across thousands of transcripts,
the reactionome approach can help to distinguish selection from neutral
drift in gene expression {[}64--66{]}. Although we cannot rule out drift
as a source of variation for individual transcripts, we would not expect
to see systematic differences in expression type categories or critical
temperature thresholds as we do here (Fig. 3, Fig. S1). Thus, our method
provides an example of how focusing on transcriptome-wide changes in
gene expression -- as opposed to identifying lists of
differentially-expressed transcripts -- can provide meaningful insight
on the process of evolution. It should be noted, however, that although
including non-linear relationships between expression and temperature
captured a significantly larger range of biologically-relevant
responses, it also led to a substantial increase in false positives.
Empirical estimation of these rates via randomization tests, combined
with robust sampling designs, can help to minimize this bias and focus
results on biologically-meaningful gene sets.

A number of caveats do apply to our work. First, species may differ in
gene expression along axes which we have not measured here, especially
temporal patterns of gene expression {[}25{]}, which could be studied in
further work. Second, the \emph{de novo} transcriptome assembly is
highly fragmented, given that all sequenced ant genomes to date have
only about 18,000 genes {[}67{]}. Although we took steps to remove
contaminants and redundant transcripts, some likely remain, in addition
to partially assembled transcripts. A genome assembly, in progress, will
help to reduce fragmentation. Third, the quality of the annotation for a
non-model system such as \emph{Aphaenogaster} is not as good as it would
be for model arthropods such as \emph{Drosophila} and \emph{Apis}.
Finally, the mapping of changes in gene expression to organismal fitness
is far from direct {[}68{]}, and large differences in patterns of gene
expression may have only small effects on fitness. In particular,
functional protein levels cannot be expected to be fully linked to mRNA
abundance due to post-transcriptional modification, regulation, mRNA
fluctuations and protein stability {[}68{]}.

Our results are congruent with evidence from other systems {[}21{]} that
thermally-stressful habitats select for investment in tolerance, whereas
organisms from less stressful environments rely on plastically-induced
resistance. Although the heat-shock response is one of the most
conserved across living organisms {[}39{]}, it is energetically
expensive, particularly under chronic stress conditions {[}69{]}. Under
such circumstances, it may be advantageous to proactively prevent
thermal damage even at the cost of reduced metabolic efficiency, either
by maintaining a higher constitutive level of chaperone proteins
{[}11{]} or by increasing the thermal stability of proteins at the
expense of catalytic activity {[}70{]}. Thus, although in the short term
increasing temperature stress leads to a quantitatively stronger induced
response, adapting to such stress over evolutionary time appears to
require a qualitative shift in mechanism of resistance that can alter
not only the magnitude, but the sign of gene expression change in
response to temperature. Whether such a shift would be possible in the
compressed time frame of projected climate change, particularly for
long-lived organisms such as ants, is likely to be critical in
determining the capacity of populations to adapt to more frequent and
long-lasting stressors.

\subsection{Conclusions}\label{conclusions}

In this work, we have brought reaction norms to the genomic era by
characterizing the thermal reactionomes of two temperate ant species,
\emph{Aphaenogaster picea} and \emph{A. carolinensis}. At least 2\% of
their transcriptomes are thermally responsive. Our results indicate that
these two ant species have different responses to thermal extremes.
\emph{A. picea} responds by increasing expression of transcripts related
to metabolism, stress response and other protective molecules, whereas
\emph{A. carolinensis} decreases expression of transcripts related to
metabolism and likely relies on other mechanisms for thermal tolerance.
The thermal reactionomes of these two species provide key insights into
the genetic basis of thermal tolerance, and a resource for the future
study of ecological adaptation in ant species. Finally, the reactionome
itself illustrates a new direction for characterizing acclimation and
adaptation in a changing climate.

\section{Methods}\label{methods}

\subsection{Samples}\label{samples}

Ants of the genus \emph{Aphaenogaster} are some of the most abundant in
eastern North America {[}71{]}, and species as well as populations
within species differ in critical maximum and minimum temperatures
{[}57{]}. Temperature is a potentially strong selective force for
ground-nesting ant populations, which must tolerate seasonally freezing
winters and hot summers. On shorter time scales, individual workers can
experience extreme thermal environments when they leave the thermally
buffered ant nest to forage for food {[}32{]}.

In fall 2012, we collected a single colony of \emph{Aphaenogaster picea}
from Molly Bog, Vermont (University of Vermont Natural Areas; 44.508° N,
-72.702° W) and a single colony of \emph{Aphaenogaster carolinensis},
part of the \emph{A. rudis} species complex {[}26{]}, from Durham, North
Carolina (36.037° N -78.874° W). These sites are centrally located
within each species' geographic range. Along the East Coast of the
United State, the distribution of \emph{A. picea} ranges from central
Maine south to northern Pennsylvania, while \emph{A. carolinensis} is
found from Pennsylvania to the Carolinas. Species identity was confirmed
with morphological characters (Bernice DeMarco, Michigan State
University). Colonies of both species were maintained in common
conditions at 25°C for 6 months prior to experimentation. Due to colony
size limitations, we were unable to determine the critical thermal
limits of these particular colonies. In summer 2013 we collected
additional colonies of \emph{Aphaenogaster} from Molly Bog, VT and North
Carolina (Duke Forest, 36.036° N, 79.077° W). We tested the upper and
lower critical thermal limits for 5 ants from each of these colonies
using a ramp of 1° C per minute, starting at 30° C, and recorded the
temperature at which the ants were no longer able to right themselves,
following the protocol of Warren \& Chick {[}57{]}.

\subsection{Common Garden Design}\label{common-garden-design}

Ideally, genetically-based variation in gene expression profiles would
be identified by comparing individuals completely reared under
common-garden conditions to eliminate environmental variation
experienced either as adults or during development. However,
\emph{Aphaenogaster} colonies are long-lived, cannot be bred under
laboratory conditions, and do not achieve complete turnover of the
workforce for at least a year or longer. Thus, as is commonly done with
other long-lived organisms {[}21, 65{]}, we exposed both colonies to
common-garden rearing conditions for six months to fully acclimate adult
workers to common temperatures. Over this time, roughly 1-2 cohorts of
new workers are expected to join each colony (\textasciitilde{}1/3 of
the total), such that the workers sampled for thermal traits and gene
expression are likely to have included a mix of adult-acclimated and
fully lab-reared individuals.

Unlike ANOVA-based experimental designs, which derive statistical power
from replication within each experimental treatment level, regression
designs have greater power when sampling additional values across the
range of the continuous predictor variable {[}72{]}. Ideally, the
treatments should be replicated at each level of the predictor variable
{[}73{]}. However, even with no replication, the regression design is
still more powerful than an ANOVA design with comparable replication,
and provides an unbiased estimator of the slope {[}72{]}. For these
reasons, we focused our sequencing efforts on maximizing the number of
temperatures at which the transcriptome was profiled, rather than on
replication at each temperature.

To limit differences in gene expression not related to the experimental
treatment (e.g.~circadian rhythm), on 12 different days we haphazardly
collected three ants from each 2012 colony at the same time of day to
minimize variation due to circadian oscillations. We measured response
to temperature with a one-hour static temperature application, which is
ecologically relevant for workers that leave the thermally-buffered nest
and are immediately exposed to ambient temperatures while foraging
{[}71{]}. Each day, the ants were placed in glass tubes immersed in a
water bath maintained at one of 12 randomly-assigned temperatures (0° to
38.5°C, in 3.5° increments) for one hour. The minimum and maximum
temperatures were selected based on previous work showing that these
temperatures are close to the critical minimum (\textasciitilde{}0°C)
and maximum (\textasciitilde{}43°C) temperatures for
\emph{Aphaenogaster} {[}57{]}, and these treatments did not cause
mortality. At the end of the hour, the ants were flash frozen in liquid
nitrogen and stored at -80°C. Thus, our reactionome characterized early,
but not late, responding genes. We extracted mRNA by homogenizing the
three pooled ants in 500 uL of RNAzol buffer with zirconium silicate
beads in a Bullet Blender (Next Advance; Averill Park, NY), followed by
RNAzol extraction (Molecular Research Center Inc; Cincinnati, OH) and
then an RNeasy micro extraction (Qiagen Inc; Valencia, CA) following the
manufacturer's instructions.

\subsection{Sequencing, assembly and
annotation}\label{sequencing-assembly-and-annotation}

For each species, the 12 samples were barcoded and sequenced in a single
lane of 2 x 100bp paired-end reads on an Illumina HiSeq 1500 yielding
200 and 160 million reads for the \emph{A. picea} and \emph{A.
carolinensis} samples respectively. Reads were filtered to remove
Illumina adapter sequences and low quality bases using the program
Trimmomatic {[}74{]}.

We assembled the sequenced reads into the full set of mRNA transcripts,
the transcriptome, for the combined data set from both species using the
Trinity \emph{de novo} transcriptome assembly program {[}75{]}. \emph{De
novo} transcriptome assembly is prone to falsely identifying alternative
transcripts and identifying inaccurate transcripts that are chimeric
(e.g.~regions of two separate transcripts that assemble into a false, or
chimeric, third transcript) {[}76{]}. We removed potentially false
transcripts by first running the program \texttt{CAP3} {[}77{]} to
cluster sequences with greater than 90\% similarity and merge
transcripts with overlaps longer than 100 bp and 98\% similar in length.
Second, we ran the program \texttt{uclust} which clusters sequences
completely contained within longer sequences at greater than 90\%
similarity (see Supplementary Methods). We used liberal values (90\%
similarity) to merge orthologous transcripts in the two species that may
not have assembled together in the initial \emph{de novo} transcriptome
assembly. To identify contaminant sequences, we screened our full
transcriptome using the program \texttt{DeconSeq} {[}78{]} with the
provided bacteria, virus, archaen and human databases of contaminants.

The Trinity \emph{de novo} transcriptome assembly for both species
assembled together included 126,172 transcripts with a total length of
100 million bp. Filtering to remove redundant or chimeric reads resulted
in an assembly with 105,536 transcripts. The total length was 63 million
bp with an N\textsubscript{50} length of 895 bp and a mean transcript
size of 593 bp. Of the 105,536 filtered transcripts, 55,432 had hits to
the NCBI-nr database. Of these, 38,711 transcripts mapped to GO terms,
1,659 transcripts were identified to an enzyme and 18,935 transcripts
mapped to a domain with \textgreater{}50\% coverage. We removed 5,675
transcripts identified as known contaminants, leaving 99,861 clean
transcripts.

We assessed the quality of the transcriptome assembly using the BUSCO
program {[}47{]} available from (\url{http://busco.ezlab.org/}). BUSCO
asseses transcriptome completeness by measuring the number of
near-universal single-copy orthologs selected from OrthoDB, using the
Arthropod database.

To determine the putative function of the transcripts, we used
functional annotation of the transcriptome assembly using the web-based
tool \texttt{FastAnnotator} {[}79{]} which annotates and classifies
transcripts by Gene Ontology (GO) term assignment, enzyme identification
and domain identification.

\subsection{Identification of thermally-responsive
transcripts}\label{identification-of-thermally-responsive-transcripts}

We quantified expression of each transcript using the program
\texttt{Sailfish} {[}48{]} and used the bias-corrected transcripts per
million (TPM) {[}80{]} as our measure of transcript expression. We
included the contaminant transcripts identified by \texttt{DeconSeq} at
the quantification stage to avoid incorrectly assigning reads to other
transcripts, but removed these from further analyses. Because
preliminary examination of the data (Supplementary Methods) indicated
that the 7°C samples may have been mis-labeled, we omitted these data
from the analysis. The expression values were highly correlated between
species at each temperature treatment (r\textsuperscript{2}
\textgreater{} 0.98) indicating that assembling the transcriptome with
data from both species was justified (Supplementary Methods).

To identify transcripts that had significant changes in expression
across the thermal gradient, we fit to each transcript an ordinary
least-squares polynomial regression model

\[ log(TPM + 1) = \beta_0 + \beta_1(species) + \beta_2(temperature) + \beta_3(temperature^2) + \beta_4(species * temperature) + \beta_5(species * temperature^2) + \epsilon \]

Temperature and species were both fixed effects, with a quadratic term
included for temperature. We used \(log(TPM + 1)\) as the response to
control for skew in the expression data. For a continuous predictor such
as temperature, this regression approach is preferred to an ANOVA
approach as it can reveal non-linear responses such as hump-shaped or
threshold effects {[}72{]}. This method is robust to over-dispersion
because we expect errors in the read count distribution {[}81{]} to be
independent with respect to temperature.

To evaluate the statistical significance of the patterns, we computed
parametric \emph{P}-values for each model and adjusted these
\emph{P}-values using the False Discovery Rate (FDR) approach of
Benjamini and Hochberg {[}82{]}. As a more stringent filter for false
positives, we then randomly re-assigned each transcript within a species
to a different temperature, fit the polynomial models as above, and
again calculated \emph{P}-values and FDR. Ideally, these randomized data
sets should not yield any significant associations. We repeated this
resampling approach 100 times, and used the 95th quantile of false
significant transcripts as the null expectation for retaining
transcripts from the true data.

Of these overall significant transcripts, we identified
thermally-responsive transcripts as the subset that had significant
\(\beta_2(temp)\), \(\beta_3(temp^2)\), \(\beta_4(species * temp)\) or
\(\beta_5(species * temp^2)\) terms after step-wise model selection by
AIC. For each thermally-responsive transcript, we predicted expression
levels using the final linear model for each species across the tested
thermal range. We used the predicted transcript expression levels to
partition transcripts for each species into the five \emph{a priori}
defined expression categories: \textbf{High} transcripts had greatest
expression at temperatures \textgreater{} 31°C, \textbf{Low} transcripts
had greatest expression at temperatures \textless{} 10°C,
\textbf{Intermediate} transcripts had greatest expression between 10 to
30°C, \textbf{Bimodal} transcripts had increased expression at both high
and low temperatures, while \textbf{NotResp} transcripts were those that
were not thermally responsive in the focal species but did respond in
the other. For the \textbf{Bimodal} group, we required that expression
at both low and high temperatures was at least one standard deviation
greater than the expression at the rearing temperature of 25°C. Because
expression category was defined by the temperature of maximal
expression, both \textbf{Low} and \textbf{High} categories were biased
toward transcripts up-regulated at that temperature extreme, but also
likely included some transcripts down-regulated at the opposing extreme.
The two categories which could unambiguously distinguish up- from
down-regulation are \emph{Bimodal} (up at both extremes) and
\emph{Intermediate} (down at both extremes).

\subsection{Statistical analyses}\label{statistical-analyses}

We used \(\chi^2\) tests to determine if the total number of responsive
transcripts, and the number of transcripts in each expression category
differed between species. To evaluate if shifts from one expression
category to another between the two species were randomly distributed,
we used the Stuart-Maxwell test of marginal homogeneity from the
\texttt{coin} package {[}83{]} in R {[}84{]} which tests if the row and
column marginal proportions are in equity.

To test whether the temperature at which thermally-responsive
transcripts were activated differs between species, we identified the
temperature at which there was the greatest change in expression for
each transcript in each species, using only the transcripts with a
significant species x temperature interaction. For upregulated
transcripts, we grouped the \textbf{High} transcripts along with the
high temperature end of the \textbf{Bimodal} transcripts, and did the
same for \textbf{Low} transcripts. We then performed a \emph{t}-test to
determine if the mean temperature of transcript activation differed
between the two species for each group. For downregulated transcripts
(i.e. \textbf{Intermediate}), we identified the greatest change in
expression for each transcript in response to both increasing
(\textgreater{} 20°C) and decreasing (\textless{} 20°C) temperatures,
and used a \emph{t}-test to compare the mean temperature of
down-regulation between species.

To test for a tradeoff between induciblity and constitutive baseline
expression between species, we fit ordinary least squares regressions
with the log ratio of relative constitutive expression as the response
variable and the log ratio of relative inducibility as the predictor
variable for \textbf{High} transcripts in \emph{A. picea} and for
\textbf{Low} transcripts in \emph{A. carolinensis}. Constitutive
expression was defined as predicted expression at 25°C, whereas
inducibility of each transcript was defined as \emph{((maximum TPM -
minimum TPM) / minimum TPM) x 100}. In addition, we used a Mann-Whitney
test to compare the baseline constitutive expression between species for
all responsive transcripts.

\subsection{Gene set enrichment
analysis}\label{gene-set-enrichment-analysis}

To describe the molecular processes involved in thermal adaptation, we
performed gene set enrichment analysis (GSEA) using the
\texttt{parentChild} algorithm {[}85{]} from the package \texttt{topGO}
{[}86{]} in R {[}84{]}. Briefly, this approach identifies GO terms that
are overrepresented in the significant transcripts relative to all GO
terms in the transcriptome, after accounting for dependencies among the
GO terms.

All analyses were performed with R 3.2 {[}84{]} and are fully
reproducible (Supplementary Methods).

\subsection{Availability of supporting
data}\label{availability-of-supporting-data}

Table S1 provides the annotation, \emph{P}-value, r\textsuperscript{2},
adjusted \emph{P}-value, and expression type for the
thermally-responsive transcripts in each species.

Table S2 provides the results of the gene set enrichment analysis,
showing the enriched gene ontology terms for each species in each
thermal response category.

The Supplementary Methods contain the detailed information on the
analysis. The reproducible and version-controlled scripts underlying the
analysis are available on GitHub
(\url{https://github.com/johnstantongeddes/ApTranscriptome}).

The Illumina short-read sequence data supporting the results of this
article are available in the NCBI Short Read Archive BioProject
repository, PRJNA260626
\url{http://www.ncbi.nlm.nih.gov/bioproject/PRJNA260626/}.

The Trinity transcriptome assembly, FastAnnotator annotation file and
Sailfish gene expression quantification files supporting the results of
this article are available from the LTER data portal, datasets hf113-38,
hf113-41, and hf113-42
(\url{http://dx.doi.org/10.6073/pasta/05ea6464df30efa2f1e2c7439366bf47}).

\subsection{Competing interests}\label{competing-interests}

The authors declare they have no competing interests.

\subsection{Authors' contributions}\label{authors-contributions}

JSG, NG and SHC designed research. JSG, ADN and LC performed research.
JSG, JV, MV and SCH analyzed data and wrote the paper. ADN and LC
performed research. JV and MV analyzed data. JSG, RD, AE, NS, NG and SHC
wrote the paper.

\section{Acknowledgements}\label{acknowledgements}

Support made possible by NSF DEB Award \#1136644 and the Vermont
Genetics Network through Grant Number 8P20GM103449 from the INBRE
Program of the National Institute of General Medical Sciences (NIGMS) of
the National Institutes of Health (NIH). Its contents are solely the
responsibility of the authors and do not necessarily represent the
official views of NIGMS or NIH.

\section{Figure Legends}\label{figure-legends}

\textbf{Figure 1}. Illustration of the patterns against temperature for
each of the four expression categories, \textbf{Bimodal}, \textbf{High},
\textbf{Intermediate} and \textbf{Low}. The fifth category of
\textbf{Not Responsive} is not shown.

\textbf{Figure 2}. Illustrations of the expected thermal response
patterns in the two species under alternative mechanistic hypotheses of
temperature adaptation. Although both temperature extremes were
investigated in a similar way, for simplicity only the response to low
temperatures is illustrated here. Each column indicates the distribution
across all response categories in \emph{A. picea}, which has a lower
CT\textsubscript{min} and is therefore better adapted to low
temperatures, for the set of transcripts identified as cold-induced
(either \textbf{High} or \textbf{Bimodal} categories) in the species
with higher CT\textsubscript{min}, \emph{A. carolinensis}, relative to
the null hypothesis of equal marginal frequencies. The dashed boxes
highlight cells that would indicate matched responses in the two
species, and the color of each cell (blue = excess, orange = deficit)
represents the deviation of the observed from expected number of
transcripts. The (A) \emph{enhanced response} hypothesis proposes that
the increase in cold tolerance in \emph{A. picea} is achieved by
amplifying existing molecular mechanisms, and thus there should be an
excess of shared response types between species. In contrast, the (B)
\emph{tolerance hypothesis} predicts that \emph{A. picea} is less
reliant on induced responses to confer cold-tolerance than \emph{A.
carolinensis}, leading to an excess of shifts from induction in \emph{A.
carolinensis} to the \textbf{Not Responsive} or down-regulation
categories in \emph{A. picea}.

\textbf{Figure 3}. Results of analysis of thermal response patterns in
the two species. The color of each cell (blue = excess, orange =
deficit) represents the deviation of the observed from the expected
number of transcripts based on hypothetical equivalence of the marginal
frequencies. The units are number of transcripts. For each temperature
extreme, the species expected to be less well adapated to that extreme
is displayed on the x-axis for the two response categories corresponding
to upregulation (\textbf{Bimodal} and \textbf{Low} for the low
temperatures, or \textbf{Bimodal} and \textbf{High} for high
temperatures). The distribution of response categories for those
transcripts in the better-adapted species is arrayed along the y-axis.
The dashed boxes indicate the matched responses (e.g. \textbf{High} -
\textbf{High}). (A) Low temperature extreme: there is an excess of
shared \textbf{Low} and \textbf{Bimodal} expression types and a bias
away from all other categories in \emph{A. picea}, consistent with the
\emph{enhanced response} hypothesis (Fig. 2). (B) High temperature
extreme: in addition to an excess of matched categories, there is an
excess of \textbf{High} and \textbf{Bimodal} transcripts in \emph{A.
picea} that are not upregulated in \emph{A. carolinensis}
(\textbf{Intermediate} and \textbf{Not Responsive}), partially
consistent with the \emph{tolerance} hypothesis. The complete set of
matched observations is shown in Fig. S1. Expression types are defined
in Table 1.

\textbf{Figure 4}. Histogram with smooth density estimate of temperature
of maximum rate of change in expression for transcripts that have (A)
increased expression at \textbf{Low} and \textbf{High} temperatures and
(B) decreased expression at \textbf{Low} and \textbf{High} temperatures.
Red bars and lines are for \emph{A. carolinensis} while blue bars and
lines are for \emph{A. picea}.

\textbf{Figure 5}. Scatterplots of log ratios of relative inducibility
to relative constitutive expression, defined as expression level at the
common rearing temperature (25°C) for (A) \textbf{High} transcripts in
\emph{A. picea} (\emph{P} \textless{} 0.001, r\textsuperscript{2} =
0.07) and (B) \textbf{Low} transcripts in \emph{A. carolinensis}
(\emph{P} \textless{} 0.001, r\textsuperscript{2} = 0.1). Blue lines and
confidence intervals are from ordinary least squares regressions.

\textbf{Figure S1.} Deviations from expected numbers of transcripts in
matched observations of transcript expression type between species
(\emph{A. carolinensis} on rows, \emph{A. picea} on columns). The color
of each cell represents the deviation of the observed from the expected
number of transcripts based on hypothetical equivalence of the marginal
frequencies (blue = excess, orange = deficit). The expression types are
\textbf{Low} transcripts that had greatest expression temperatures
\textless{} 10°C, \textbf{Intermediate} transcripts with greatest
expression between 10 and 30°C, \textbf{High} transcripts that had
greatest expression at temperatures \textgreater{} 31°, \textbf{Bimodal}
transcripts with increased expression at both high and low temperatures,
and \textbf{Not Responsive} transcripts that were not thermally
responsive in that species.

\newpage

\section{Tables}\label{tables}

\begin{longtable}[c]{@{}cccccc@{}}
\caption{Table of the number of thermally-responsive transcripts by
expression type for \emph{A. carolinensis} and \emph{A. picea}.
\textbf{Low} are transcripts with increased expression at low
temperatures (\textless{} 10°C), \textbf{Intermediate} are transcripts
with maximum expression between 10 - 30°C, \textbf{High} are transcripts
with increased expression at high temperatures (\textgreater{} 31°C),
\textbf{Bimodal} are transcripts with increased expression at both low
and high temperatures, while \textbf{NotResp} are transcripts that are
not thermally responsive in one species but are in the other
species.}\tabularnewline
\toprule
~ & Low & Intermediate & High & Bimodal & NotResp\tabularnewline
\midrule
\endfirsthead
\toprule
~ & Low & Intermediate & High & Bimodal & NotResp\tabularnewline
\midrule
\endhead
\textbf{A. picea} & 1,193 & 249 & 248 & 278 & 110\tabularnewline
\textbf{A. carolinensis} & 920 & 680 & 232 & 117 & 129\tabularnewline
\bottomrule
\end{longtable}

\begin{figure}[htbp]
\centering
\includegraphics{../results/expression_types_Fig1.png}
\caption{Fig. 1}
\end{figure}

\begin{figure}[htbp]
\centering
\includegraphics{../results/predictions_Fig2.png}
\caption{Fig. 2}
\end{figure}

\begin{figure}[htbp]
\centering
\includegraphics{../results/matched_observations_Fig3.png}
\caption{Fig. 3}
\end{figure}

\begin{figure}[htbp]
\centering
\includegraphics{../results/crit_temp_regulation_Fig4.png}
\caption{Fig. 4}
\end{figure}

\begin{figure}[htbp]
\centering
\includegraphics{../results/genetic_assimilation_Fig5.png}
\caption{Fig. 5}
\end{figure}

\newpage

\section*{References}\label{references}
\addcontentsline{toc}{section}{References}

1. Brown JH, Gillooly JF, Allen AP, Savage VM, West GB: \textbf{Toward a
metabolic theory of ecology}. \emph{Ecology} 2004,
\textbf{85}:1771--1789.

2. Kingsolver JG: \textbf{The well-temperatured biologist}.
\emph{American Naturalist} 2009, \textbf{174}:755--768.

3. Deutsch CA, Tewksbury JJ, Huey RB, Sheldon KS, Ghalambor CK, Haak DC,
Martin PR: \textbf{Impacts of climate warming on terrestrial ectotherms
across latitude}. \emph{Proceedings of the National Academy of Sciences}
2008, \textbf{105}:6668--6672.

4. Kingsolver JG, Diamond SE, Buckley LB: \textbf{Heat stress and the
fitness consequences of climate change for terrestrial ectotherms}.
\emph{Functional Ecology} 2013, \textbf{27}:1415--1423.

5. Sunday JM, Bates AE, Kearney MR, Colwell RK, Dulvy NK, Longino JT,
Huey RB: \textbf{Thermal-safety margins and the necessity of
thermoregulatory behavior across latitude and elevation}.
\emph{Proceedings of the National Academy of Sciences} 2014:201316145.

6. Huey RB, Kingsolver JG: \textbf{Evolution of thermal sensitivity of
ectotherm performance}. \emph{Trends in Ecology \& Evolution} 1989,
\textbf{4}:131--135.

7. Richter K, Haslbeck M, Buchner J: \textbf{The heat shock response:
Life on the verge of death}. \emph{Molecular Cell} 2010,
\textbf{40}:253--266.

8. Angilletta MJ, Wilson RS, Navas CA, James RS: \textbf{Tradeoffs and
the evolution of thermal reaction norms}. \emph{Trends in Ecology \&
Evolution} 2003, \textbf{18}:234--240.

9. Cowles RB: \textbf{Possible implications of reptilian thermal
tolerance}. \emph{Science} 1939, \textbf{90}:465--466.

10. Hoffmann AA, Chown SL, Clusella-Trullas S: \textbf{Upper thermal
limits in terrestrial ectotherms: How constrained are they?}
\emph{Functional Ecology} 2013, \textbf{27}:934--949.

11. Krebs R, Loeschcke V: \textbf{Estimating heritability in a threshold
trait: Heat-shock tolerance in drosophila buzzatii}. \emph{Heredity}
1997, \textbf{79}:252--259.

12. Kellermann V, Overgaard J, Hoffmann AA, Fl{ø}jgaard C, Svenning J-C,
Loeschcke V: \textbf{Upper thermal limits of \emph{drosophila} are
linked to species distributions and strongly constrained
phylogenetically}. \emph{Proceedings of the National Academy of
Sciences} 2012, \textbf{109}:16228--16233.

13. Krebs RA, Feder ME, Lee J: \textbf{Heritability of expression of the
70KD heat-shock protein in \emph{drosophila melanogaster} and its
relevance to the evolution of thermotolerance}. \emph{Evolution} 1998,
\textbf{52}:841--847.

14. Williams BR, Van Heerwaarden B, Dowling DK, Sgr{ò} CM: \textbf{A
multivariate test of evolutionary constraints for thermal tolerance in
\emph{drosophila melanogaster}}. \emph{Journal of Evolutionary Biology}
2012, \textbf{25}:1415--1426.

15. Morgan TJ, Mackay TFC: \textbf{Quantitative trait loci for
thermotolerance phenotypes in \emph{drosophila melanogaster}}.
\emph{Heredity} 2006, \textbf{96}:232--242.

16. Takahashi KH, Okada Y, Teramura K: \textbf{Genome-wide deficiency
screen for the genomic regions responsible for heat resistance in
\emph{drosophila melanogaster}}. \emph{BMC Genetics} 2011,
\textbf{12}:57.

17. Hoffmann AA, Willi Y: \textbf{Detecting genetic responses to
environmental change}. \emph{Nature Reviews Genetics} 2008,
\textbf{9}:421--432.

18. Somero GN: \textbf{Comparative physiology: A ``crystal ball'' for
predicting consequences of global change}. \emph{American Journal of
Physiology - Regulatory, Integrative and Comparative Physiology} 2011,
\textbf{301}:R1--R14.

19. Meyer E, Aglyamova GV, Matz MV: \textbf{Profiling gene expression
responses of coral larvae (\emph{acropora millepora}) to elevated
temperature and settlement inducers using a novel RNA-seq procedure}.
\emph{Molecular Ecology} 2011, \textbf{20}:3599--616.

20. Teets NM, Peyton JT, Colinet H, Renault D, Kelley JL, Kawarasaki Y,
Lee RE, Denlinger DL: \textbf{Gene expression changes governing extreme
dehydration tolerance in an antarctic insect}. \emph{Proceedings of the
National Academy of Sciences} 2012, \textbf{109}:20744--9.

21. Barshis DJ, Ladner JT, Oliver TA, Seneca FO, Traylor-Knowles N,
Palumbi SR: \textbf{Genomic basis for coral resilience to climate
change}. \emph{Proceedings of the National Academy of Sciences} 2013,
\textbf{110}:1387--1392.

22. O'Neil ST, Dzurisin JDK, Williams CM, Lobo NF, Higgins JK, Deines
JM, Carmichael RD, Zeng E, Tan JC, Wu GC, Emrich SJ, Hellmann JJ:
\textbf{Gene expression in closely related species mirrors local
adaptation: Consequences for responses to a warming world}.
\emph{Molecular Ecology} 2014, \textbf{23}:2686--2698.

23. Gomulkiewicz R, Kirkpatrick M: \textbf{Quantitative genetics and the
evolution of reaction norms}. \emph{Evolution} 1992,
\textbf{46}:390--411.

24. Murren CJ, Maclean HJ, Diamond SE, Steiner UK, Heskel MA, Handelsman
CA, Ghalambor CK, Auld JR, Callahan HS, Pfennig DW, Relyea RA,
Schlichting Carl D., Kingsolver J: \textbf{Evolutionary change in
continuous reaction norms.} \emph{The American Naturalist} 2014,
\textbf{183}:453--467.

25. S{ø}rensen JG, Nielsen MM, Kruh{ø}ffer M, Justesen J, Loeschcke V:
\textbf{Full genome gene expression analysis of the heat stress response
in \emph{drosophila melanogaster}}. \emph{Cell Stress \& Chaperones}
2005, \textbf{10}:312--328.

26. Umphrey G: \textbf{Morphometric discrimination among sibling species
in the \emph{fulva - rudis - texana} complex of the ant genus
aphaenogaster}. \emph{Canadian Journal of Zoology} 1996,
\textbf{74}:528--559.

27. DeMarco B, Cognato A: \textbf{A multiple-gene phylogeny reveals
polyphyly among eastern north american \emph{aphaenogaster} species}.
\emph{Zoologica} 2015, \textbf{doi:10.1111/zsc.12168}.

28. King JR, Warren RJ, Bradford MA: \textbf{Social insects dominate
eastern US temperate hardwood forest macroinvertebrate communities in
warmer regions}. \emph{PLoS ONE} 2013, \textbf{8}:e75843.

29. Ness JH, Morin DF, Giladi I: \textbf{Uncommon specialization in a
mutualism between a temperate herbaceous plant guild and an ant: Are
\emph{aphaenogaster} ants keystone mutualists?} \emph{Oikos} 2009,
\textbf{118}:1793--1804.

30. Zelikova TJ, Sanders NJ, Dunn RR: \textbf{The mixed effects of
experimental ant removal on seedling distribution, belowground
invertebrates, and soil nutrients}. \emph{Ecosphere} 2011,
\textbf{2}:art63.

31. Rodriguez-Cabal MA, Stuble KL, Gu{é}nard B, Dunn RR, Sanders NJ:
\textbf{Disruption of ant-seed dispersal mutualisms by the invasive
asian needle ant (\emph{pachycondyla chinensis})}. \emph{Biological
Invasions} 2012, \textbf{14}:557--565.

32. Diamond SE, Nichols LM, McCoy N, Hirsch C, Pelini SL, Sanders NJ,
Ellison AM, Gotelli NJ, Dunn RR: \textbf{A physiological trait-based
approach to predicting the responses of species to experimental climate
warming}. \emph{Ecology} 2012, \textbf{93}:2305--2312.

33. Toro ID, Ribbons RR, Pelini SL: \textbf{The little things that run
the world revisited: A review of ant-mediated ecosystem services and
disservices (hymenoptera: Formicidae)}. \emph{Myrmecological News} 2012,
\textbf{17}:133--146.

34. Pelini SL, Diamond SE, Maclean HJ, Ellison AM, Gotelli NJ, Sanders
NJ, Dunn RR: \textbf{Common garden experiments reveal uncommon responses
across temperatures, locations, and species of ants}. \emph{Ecology and
Evolution} 2012, \textbf{2}:3009--15.

35. Sharon B, Stuble KL, Lessard J-P, Dunn RR, Adler FR, Sanders NJ:
\textbf{Predicting future coexistence in a north american ant
community}. \emph{Ecology and Evolution} 2014, \textbf{4}:1804--1819.

36. Hijmans R, Cameron S, Parra J, Jones P, Jarvis A: \textbf{Very high
resolution interpolated climate surfaces of global land areas}.
\emph{International Journal of Climatology} 2005,
\textbf{25}:1965--1978.

37. Hofmann GE, Somero GN: \textbf{Interspecific variation in thermal
denaturation of proteins in the congeneric mussels mytilus trossulus and
m. galloprovincialis: Evidence from the heat-shock response and protein
ubiquitination}. \emph{Marine Biology} 1996, \textbf{126}:65--75.

38. Feder ME, Hofmann GE: \textbf{Heat-shock proteins, molecular
chaperones, and the stress response: Evolutionary and ecological
physiology}. \emph{Annual Review of Physiology} 1999,
\textbf{61}:243--282.

39. K{ü}ltz D: \textbf{Molecular and evolutionary basis of the cellular
stress response}. \emph{Annual Review of Physiology} 2005,
\textbf{67}:225--257.

40. Fields PA: \textbf{Protein function at thermal extremes: Balancing
stability and flexibility}. \emph{Comparative Biochemistry and
Physiology Part A: Molecular \& Integrative Physiology} 2001,
\textbf{129}:417--431.

41. Lockwood BL, Somero GN: \textbf{Functional determinants of
temperature adaptation in enzymes of cold- versus warm-adapted mussels
(genus \emph{mytilus})}. \emph{Molecular Biology and Evolution} 2012,
\textbf{29}:3061--3070.

42. Gordon DM: \textbf{The rewards of restraint in the collective
regulation of foraging by harvester ant colonies}. \emph{Nature} 2013,
\textbf{498}:91--93.

43. Neelakanta G, Hudson AM, Sultana H, Cooley L, Fikrig E:
\textbf{Expression of \emph{ixodes scapularis} antifreeze glycoprotein
enhances cold tolerance in \emph{drosophila melanogaster}}. \emph{PLoS
ONE} 2012, \textbf{7}:e33447.

44. Franssen SU, Bergmann N, Winters G, Klostermeier UC, Rosenstiel P,
Bornberg-Bauer E, Reusch TBH: \textbf{Transcriptomic resilience to
global warming in the seagrass zostera marina, a marine foundation
species}. \emph{Proceedings of the National Academy of Sciences} 2011,
\textbf{108}:19276--19281.

45. Waddington C: \textbf{Genetic assimilation of an acquired
character}. \emph{Evolution} 1953, \textbf{7}:118--126.

46. Sikkink KL, Reynolds RM, Ituarte CM, Cresko WA, Phillips PC:
\textbf{Rapid evolution of phenotypic plasticity and shifting thresholds
of genetic assimilation in the nematode
\textless{}i\textgreater{}Caenorhabditis
remanei\textless{}i\textgreater{}}. \emph{G3: Genes Genomes Genetics}
2014, \textbf{4}:1103--1112.

47. Simao FA, Waterhouse RM, Ioannidis P, Kriventseva EV: \textbf{BUSCO:
Assessing genome assembly and annotation completeness with single-copy
orthologs}. \emph{Bioinformatics} 2015,
\textbf{10.1093/bioinformatics/btv351}.

48. Patro R, Mount SM, Kingsford C: \textbf{Sailfish enables
alignment-free isoform quantification from RNA-seq reads using
lightweight algorithms}. \emph{Nature Biotechnology} 2014,
\textbf{32}:462--464.

49. Vijay N, Poelstra JW, K{ü}nstner A, Wolf JBW: \textbf{Challenges and
strategies in transcriptome assembly and differential gene expression
quantification. a comprehensive in silico assessment of RNA-seq
experiments}. \emph{Molecular Ecology} 2013, \textbf{22}:620--634.

50. Ohtsu T, Kimura MT, Katagiri C: \textbf{How \emph{drosophila}
species acquire cold tolerance}. \emph{Eur J Biochem} 1998,
\textbf{252}:608--611.

51. Denlinger DL: \textbf{Regulation of diapause}. \emph{Annual Review
of Entomology} 2002, \textbf{47}:93--122.

52. Ghosh K, Dill K: \textbf{Cellular proteomes have broad distributions
of protein stability}. \emph{Biophysical Journal} 2010,
\textbf{99}:3996--4002.

53. Teets NM, Peyton JT, Ragland GJ, Colinet H, Renault D, Hahn DA,
Denlinger DL: \textbf{Combined transcriptomic and metabolomic approach
uncovers molecular mechanisms of cold tolerance in a temperate flesh
fly}. \emph{Physiological Genomics} 2012, \textbf{44}:764--777.

54. Vesala L, Salminen T, Laiho A, Hoikkala A, Kankare M: \textbf{Cold
tolerance and cold-induced modulation of gene expression in two
\emph{drosophila virilis} group species with different distributions:
Cold-induced changes in gene expression}. \emph{Insect Molecular
Biology} 2012, \textbf{21}:107--118.

55. Addo-Bediako A, Chown SL, Gaston KJ: \textbf{Metabolic cold
adaptation in insects: A large-scale perspective}. \emph{Functional
Ecology} 2002, \textbf{16}:332--338.

56. Kelly MW, Grosberg Richard K., Sanford E: \textbf{Trade-offs,
geography, and limits to thermal adaptation in a tide pool copepod.}
\emph{The American Naturalist} 2013, \textbf{181}:846--854.

57. Warren RJ, Chick L: \textbf{Upward ant distribution shift
corresponds with minimum, not maximum, temperature tolerance}.
\emph{Global Change Biology} 2013, \textbf{19}:2082--2088.

58. Addo-Bediako A, Chown SL, Gaston KJ: \textbf{Thermal tolerance,
climatic variability and latitude}. \emph{Proceedings of The Royal
Society B: Biological Sciences} 2000, \textbf{267}:739--745.

59. Hodgins-Davis A, Adomas AB, Warringer J, Townsend JP:
\textbf{Abundant gene-by-environment interactions in gene expression
reaction norms to copper within \emph{saccharomyces cerevisiae}}.
\emph{Genome Biology and Evolution} 2012, \textbf{4}:1061--1079.

60. Aubin-Horth N, Renn SCP: \textbf{Genomic reaction norms: Using
integrative biology to understand molecular mechanisms of phenotypic
plasticity}. \emph{Molecular Ecology} 2009, \textbf{18}:3763--3780.

61. Gonz{á}lez E, Joly S: \textbf{Impact of RNA-seq attributes on false
positive rates in differential expression analysis of de novo assembled
transcriptomes}. \emph{BMC Research Notes} 2013, \textbf{6}:503.

62. Sarup P, S{ø}rensen JG, Kristensen TN, Hoffmann AA, Loeschcke V,
Paige KN, S{ø}rensen P: \textbf{Candidate genes detected in
transcriptome studies are strongly dependent on genetic background}.
\emph{PLoS ONE} 2011, \textbf{6}:e15644.

63. Pavlidis P, Jensen JD, Stephan W, Stamatakis A: \textbf{A critical
assessment of storytelling: Gene ontology categories and the importance
of validating genomic scans}. \emph{Molecular Biology and Evolution}
2012, \textbf{29}:3237--3248.

64. Khaitovich P, Weiss G, Lachmann M, Hellmann I, Enard W, Muetzel B,
Wirkner U, Ansorge W, P{ä}{ä}bo S: \textbf{A neutral model of
transcriptome evolution.} \emph{PLoS Biology} 2004, \textbf{2}:E132.

65. Whitehead A, Crawford DL: \textbf{Neutral and adaptive variation in
gene expression}. \emph{Proceedings of the National Academy of Sciences}
2006, \textbf{103}:5425--5430.

66. Ogasawara O, Okubo K: \textbf{On theoretical models of gene
expression evolution with random genetic drift and natural selection.}
\emph{PLoS ONE} 2009, \textbf{4}:e7943.

67. Gadau J, Helmkampf M, Nygaard S, Roux J, Simola DF, Smith CR, Suen
G, Wurm Y, Smith CD: \textbf{The genomic impact of 100 million years of
social evolution in seven ant species.} \emph{Trends in Genetics} 2012,
\textbf{28}:14--21.

68. Feder ME, Walser J-C: \textbf{The biological limitations of
transcriptomics in elucidating stress and stress responses}.
\emph{Journal of Evolutionary Biology} 2005, \textbf{18}:901--910.

69. Hoekstra LA, Montooth KL: \textbf{Inducing extra copies of the hsp70
gene in \emph{drosophila melanogaster} increases energetic demand}.
\emph{BMC Evolutionary Biology} 2013, \textbf{13}:1--11.

70. Arnold FH, Wintrode PL, Miyazaki K, Gershenson A: \textbf{How
enzymes adapt: Lessons from directed evolution}. \emph{Trends in
Biochemical Sciences} 2001, \textbf{26}:100--106.

71. Lubertazzi D: \textbf{The biology and natural history of
\emph{aphaenogaster rudis}}. \emph{Psyche: A Journal of Entomology}
2012, \textbf{2012}:1--11.

72. Gotelli NJ, Ellison AM: \emph{A Primer of Ecological Statistics}.
2nd edition. Sunderland, MA: Sinauer Associates, Inc; 2012.

73. Cottingham KL, Lennon JT, Brown BL: \textbf{Knowing when to draw the
line: Designing more informative ecological experiments}.
\emph{Frontiers in Ecology and the Environment} 2005,
\textbf{3}:145--152.

74. Lohse M, Bolger AM, Nagel A, Fernie AR, Lunn JE, Stitt M, Usadel B:
\textbf{RobiNA: A user-friendly, integrated software solution for
RNA-seq-based transcriptomics}. \emph{Nucleic Acids Research} 2012,
\textbf{40}:W622--W627.

75. Grabherr MG, Haas BJ, Yassour M, Levin JZ, Thompson D a, Amit I,
Adiconis X, Fan L, Raychowdhury R, Zeng Q, Chen Z, Mauceli E, Hacohen N,
Gnirke A, Rhind N, Palma F di, Birren BW, Nusbaum C, Lindblad-Toh K,
Friedman N, Regev A: \textbf{Full-length transcriptome assembly from
RNA-seq data without a reference genome.} \emph{Nature Biotechnology}
2011, \textbf{29}:644--52.

76. Yang Y, Smith SA: \textbf{Optimizing de novo assembly of short-read
RNA-seq data for phylogenomics}. \emph{BMC Genomics} 2013,
\textbf{14}:328.

77. Huang X, Madan A: \textbf{CAP3: A DNA sequence assembly program}.
\emph{Genome Research} 1999, \textbf{9}:868--877.

78. Schmieder R, Edwards R: \textbf{Fast identification and removal of
sequence contamination from genomic and metagenomic datasets}.
\emph{PLoS ONE} 2011, \textbf{6}:e17288.

79. Chen T-W, Gan R-CR, Wu TH, Huang P-J, Lee C-Y, Chen Y-YM, Chen C-C,
Tang P: \textbf{FastAnnotator- an efficient transcript annotation web
tool}. \emph{BMC Genomics} 2012, \textbf{13}(Suppl 7):S9.

80. Wagner GP, Kin K, Lynch VJ: \textbf{Measurement of mRNA abundance
using RNA-seq data: RPKM measure is inconsistent among samples}.
\emph{Theory in Biosciences} 2012, \textbf{131}:281--285.

81. Anders S, Huber W: \textbf{Differential expression analysis for
sequence count data}. \emph{Genome Biology} 2010, \textbf{11}:R106.

82. Benjamini Y, Hochberg Y: \textbf{Controlling the false discovery
rate: A practical and powerful approach to multiple testing}.
\emph{Journal of the Royal Statistical Society Series B
(Methodological)} 1995, \textbf{57}:289--300.

83. Hothorn T, Hornik K, Mark van de Wiel, Zeileis A:
\textbf{Implementing a class of permutation tests: The coin package}.
\emph{Journal of Statistical Software} 2008, \textbf{28}:1--23.

84. R Core Team: \textbf{R: A language and environment for statistical
computing}. 2013.

85. Grossmann S, Bauer S, Robinson PN, Vingron M: \textbf{Improved
detection of overrepresentation of gene-ontology annotations with
parent--child analysis}. \emph{Bioinformatics} 2007,
\textbf{23}:3024--3031.

86. Alexa A, Rahnenf{ü}hrer J, Lengauer T: \textbf{Improved scoring of
functional groups from gene expression data by decorrelating GO graph
structure}. \emph{Bioinformatics} 2006, \textbf{22}:1600--1607.

\end{document}

\documentclass[letterpaper]{article}

\usepackage{graphicx} % support the \includegraphics command and options
\usepackage{color} % color!
\usepackage[hidelinks]{hyperref} % allow html links, hide defaul link box

\usepackage[textheight=8in]{geometry} % reduce textbox height to fit header (next section) and change page margins to 1 in

\usepackage{parskip} % change format to blank line between paragraphs instead of tab

\usepackage{fancyhdr}

\fancypagestyle{first}{%
  \fancyhf{}
  \setlength{\headheight}{40pt}
  \fancyhead[L]{\includegraphics[scale=0.1]{UVM_Official_Logo.jpg}} 
  \fancyhead[R]{\small Department of Biology \\102  Marsh Life Sciences \\ Burlington, VT 05405 }
  \renewcommand{\headrulewidth}{0.3pt}
  \renewcommand{\footrulewidth}{0pt}}
\pagestyle{first}

\fancypagestyle{plain}{%
  \fancyhf{}
  \setlength{\headheight}{0pt}
  \fancyhead[L]{ } 
  \fancyhead[R]{\small John Stanton-Geddes} 
  \fancyfoot[C]{\thepage}
  \renewcommand{\headrulewidth}{0.4pt}
  \renewcommand{\footrulewidth}{0pt}}
\pagestyle{plain}


\begin{document}

\thispagestyle{first}

\noindent{Dear \emph{PNAS} Editorial Board,}

We are excited to submit our manuscript \emph{Thermal
reactionomes reveal adaptive responses to thermal extremes in warm and
cold-climate ant species} for consideration as a Research Report in
\emph{The Proceedings of the National Academy of Sciences}. There are
several aspects of the work that we think make it worthy of publication
in \emph{PNAS}.

In face of contemporary climate change, many species are predicted to be at 
risk of extinction. However, there is also evidence that species can physiologically
acclimate and rapidly adapt to changing environmental conditions, potentially
rescuing them from extinction. How these two processes are related at the molecular
level is poorly understood, hampered by the limitations inherent in extrapolating 
complex responses and functional interactions from the relatively simple experimental
design of most gene expression experiments to date. 

In this study, we resolve this problem by extending the reaction
norm approach to the study of gene expression. We define the
\emph{reactionome} as a characterization of the reaction norm for all
genes in an organism's transcriptome across an environmental gradient. 
We believe that this approach is an important contribution as it allows us
to make quantitative comparisons of plasticity in gene expression between
species. Importantly, this analysis does not depend on the functional 
annotation of the genes, and thus is robust to criticisms of "storytelling" 
in RNAseq studies based on annotation of differentially expressed 
genes (Pavlidis et al. 2012 MBE). 

By applying the reactionome approach to two related ant species,
\emph{Aphaenogaster picea} and \emph{A. carolinensis}, from colder and
warmer climates, respectively, we gain insight on the relationship
between short-term physiological acclimation and longer-term adaptive
change in response to thermal extremes. In particular, we find that while
extensions of lower critical thermal limits involve enhancing existing plasticity,
adaptation to chronic high temperatures involves a shift from induced resistance
toward novel tolerance mechanisms, which may limit adaptive potential 
in the face of rapidly rising temperatures associated with global climate change.

We believe that the data, including the first transcriptome of a
temperate ant species, analyses and results will be of interest to
multiple audiences including ecologists working on species
distributions, physiologists interested in thermal tolerance and
evolutionary geneticists interested in the evolution of gene expression.

This manuscript is most appropriate for the major classification
BIOLOGICAL SCIENCES, minor classification Ecology.

\textbf{Suggested Editors:} Douglas J. Futuyma, David L. Denlinger, David M. Hillis

\textbf{Suggested NAS Members:} Trudy F. Mackay, Nevo Eviata, George N. Somero

\textbf{Suggested reviewers:}  Dr. Nicholas Teets, Postdoc, University of Florida; Dr. Vanessa Kellerman, Postdoc, Monash University; Dr. Pavlos Pavlidis, Postdoc, Heidelberg Institute of Theoretical Studies; Dr. Stephen Palumbi, Professor, Hopkins Marine Station, Stanford University; Dr. Joel Kingsolver, Professor, University of North Carolina

\noindent{Thank you for consideration of our work,} \\
\includegraphics[scale=0.8]{JSG_signature.png}  \\
Dr. John Stanton-Geddes on behalf of all the authors

\end{document}